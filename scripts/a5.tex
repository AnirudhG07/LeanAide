\documentclass{amsart}
\theoremstyle{plain}
\newtheorem{question}{Question}

\begin{document}

\begin{question}
    Find all $2 \times 2$ matrices with rational entries such that $A^5 = I$, where $I$ is the $2 \times 2$ identity matrix.
\end{question}

\begin{proof}
    A $2 \times 2$ matrix $A$ over the rationals is of the form

\[
A = \begin{pmatrix}
a & b \\
c & d 
\end{pmatrix},
\]

where $a, b, c, d \in \mathbb{Q}$. We want to find all such $A$ such that $A^5 = I$, where $I$ is the $2 \times 2$ identity matrix. 

We can start by noting that the minimal polynomial of such an $A$ must divide $x^5 - 1 = 0$. Over the rationals, this polynomial factors into irreducible factors as

\[
x^5 - 1 = (x - 1)(x^4 + x^3 + x^2 + x + 1).
\]

So the minimal polynomial of $A$ must either be $x - 1$ or $x^4 + x^3 + x^2 + x + 1 = 0$. However, the second polynomial has degree $4$ and the minimal polynomial of a $2 \times 2$ matrix can have degree at most $2$, so $A$ must satisfy $x - 1 = 0$. That is, $A = I$. 

To verify our claim, we note that if $A = I$, then indeed $A^5 = I^5 = I$. So the only $2 \times 2$ matrix over the rationals such that $A^5 = I$ is $A = I$.
\end{proof}

\begin{proof}The condition $A^5 = I$ implies that the minimal polynomial of $A$ divides $x^5 - 1 = 0$. The factorization of $x^5 - 1$ over the rational numbers is $(x-1)(x^4+x^3+x^2+x+1) = 0$. 

The minimal polynomial of a $2 \times 2$ matrix is at most of degree 2. Therefore, the minimal polynomial of $A$ can either be $x-1=0$, $x^2-1=0$, $x^2+x+1=0$, or $x^4+x^3+x^2+x+1=0$.

1. If the minimal polynomial is $x-1=0$, then $A=I$ is the only solution.

2. If the minimal polynomial is $x^2-1=0$, then the solutions correspond to matrices $A$ which are similar to the diagonal matrix with entries $1,-1$. The matrices similar to a diagonal matrix with entries $1,-1$ are of the form $\begin{pmatrix} a & b \ c & -a \end{pmatrix}$ such that $a^2 - bc = 1$.

3. If the minimal polynomial is $x^2+x+1=0$, then the solutions correspond to matrices $A$ which are similar to the diagonal matrix with entries $\omega, \omega^2$, where $\omega$ is a primitive cube root of unity. However, these matrices cannot have rational entries.

4. If the minimal polynomial is $x^4+x^3+x^2+x+1=0$, it cannot be the minimal polynomial of a $2 \times 2$ matrix since its degree is greater than 2.

In summary, the solutions are all $2 \times 2$ matrices of the form $\begin{pmatrix} a & b \ c & -a \end{pmatrix}$ such that $a^2 - bc = 1$ and the $2 \times 2$ identity matrix. This solution is complete.
\end{proof}

\begin{proof}To find all 2x2 matrices with rational entries such that $A^5 = I$, where $I$ is the 2x2 identity matrix, we need to consider the cyclic subgroup generated by A under matrix multiplication, which has order 5.

Firstly, observe that the minimal polynomial of such a matrix $A$ must divide $x^5 - 1 = 0$.
Decomposing this into irreducible polynomials over the rationals (using the fact that it is a cyclotomic polynomial), we get $x^5 - 1 = (x-1)(x^4 + x^3 + x^2 + x + 1)$.

The roots of $x-1$ are 1, which corresponds to the identity matrix. The roots of $x^4 + x^3 + x^2 + x + 1$ are the 4 complex 5th roots of unity which are not 1, namely $e^{2\pi i / 5}, e^{4\pi i / 5}, e^{6\pi i / 5}$, and $e^{8\pi i / 5}$.

Since the minimal polynomial of a 2x2 matrix must be of degree at most 2, the minimal polynomial of $A$ can be either $(x-1)$, $(x - e^{2\pi i / 5})(x - e^{8\pi i / 5})$, or $(x - e^{4\pi i / 5})(x - e^{6\pi i / 5})$. The case of $(x-1)$ corresponds to the identity matrix.

In the other two cases, the matrices corresponding will be similar to the matrix:

$A = \begin{bmatrix}
a & -b\
b & a
\end{bmatrix}$

for the two cases where (a, b) are elements of either $(cos(2\pi/5), sin(2\pi/5))$ or $(cos(4\pi/5), sin(4\pi/5))$. The other two roots are negatives of these two and will give the same matrices (up to similarity).

To prove this, note that any 2x2 matrix with a minimal polynomial of degree 2 is similar to a matrix in this form, and that a matrix in this form has the desired properties. Also, any 2x2 matrix satisfying $A^5 = I$ has a minimal polynomial dividing $x^5 - 1$, and so its minimal polynomial must be one of the three cases we have listed.

So finally, the matrices $A$ in question are the $2 \times 2$ matrices with rational entries which are similar to either the identity matrix, or one of the two matrices corresponding to $cos(2\pi/5)$, $sin(2\pi/5)$ or $cos(4\pi/5)$, $sin(4\pi/5)$.
\end{proof}


\begin{question}
    Prove that the polynomial $x^4 + x^3 + x^2 + x + 1$ is irreducible over the rationals.
\end{question}

\begin{proof}To show that the polynomial $x^4 + x^3 + x^2 + x + 1$ is irreducible over the rationals, we can use the following strategy. 

First, we make a substitution $y = x+1$, so that the polynomial becomes $y^4 - C$ where $C$ is a constant (specifically, $C = 5 + 4y + 6y^2 + 4y^3$). 

Now it's clear that if the polynomial were reducible, its factors would have to be of the form $(y^2 + ay + b)(y^2 + cy + d)$ for some rational numbers $a, b, c, d$. But then we would have $ac=0$, (since $y^4$ has no $y^3$ term), which implies that either $a=0$ or $c=0$.

However, if either $a=0$ or $c=0$, then the constant term of one of the factors would have to be a square root of $C$, but $C$ is not a perfect square, so this is impossible. Hence the polynomial is irreducible over the rationals. 

The underlying principle here is Eisenstein's Criterion, which states that a polynomial is irreducible if there exists a prime number $p$ such that $p$ divides all coefficients of the polynomial except the leading one, $p^2$ does not divide the constant term, and the leading coefficient is not divisible by $p$. In this case, the prime number is $p=1$, and the coefficients of the polynomial are all 1.\end{proof}
\begin{proof}There are many ways to show that a polynomial is irreducible over the rationals, but a particularly useful one here is a concept of Eisenstein's Criterion. However, Eisenstein's Criterion is not directly applicable for this polynomial. 

One way we can apply it is through what's called a change of variables. We need to find a suitable substitution that transforms the polynomial into a form where Eisenstein's Criterion applies. 

Let's let $y = x+1$. Then we have
$y^4 + y^3 + y^2 + y + 1 = (x+1)^4 + (x+1)^3 + (x+1)^2 + (x+1) + 1 = x^4 + 5x^3 + 10x^2 + 10x + 5.$ 

Now we can apply Eisenstein's Criterion with $p = 5$ to conclude that $x^4 + 5x^3 + 10x^2 + 10x + 5$ is irreducible over the rationals.

Because the roots of the original polynomial $x^4 + x^3 + x^2 + x + 1$ are precisely the roots of $x^4 + 5x^3 + 10x^2 + 10x + 5$ decreased by 1, we can conclude that the original polynomial is also irreducible over the rationals.

Eisenstein's Criterion states that if you have a polynomial 
\[a_nx^n + a_{n-1}x^{n-1} + ... + a_1x + a_0\]
and there exists a prime number $p$ such that:
1. $p$ divides $a_i$ for all $i < n$,
2. $p$ does not divide $a_n$,
3. $p^2$ does not divide $a_0$,

then the polynomial is irreducible over the rationals.\end{proof}
\begin{proof}To prove that the polynomial $x^4 + x^3 + x^2 + x + 1$ is irreducible over the rationals, one approach is by using a substitution and then applying Eisenstein's Criterion, a well-known theorem in number theory that provides a sufficient condition for a polynomial to be irreducible over the rational numbers.

Let the polynomial be denoted as $f(x) = x^4 + x^3 + x^2 + x + 1$. Now consider the substitution $x = y-1$, that is, replace $x$ with $(y-1)$ in the polynomial given. This gives us a new polynomial:

$g(y) = (y-1)^4 + (y-1)^3 + (y-1)^2 + (y-1) + 1 = y^4 - 4y^3 + 6y^2 - 4y + 1$.

Now, observe that all the coefficients of $g$ are integers, and every coefficient of a term whose degree is less than 4 is divisible by 4, but $4^2$ does not divide the constant term 1. Furthermore, the constant term 1 is a prime number. 

Therefore, by Eisenstein's Criterion (with prime number p=4), the polynomial $g(y)$ is irreducible over Q.

Finally, because $f(x)$ and $g(y)$ are related by a simple substitution $(y=x+1)$, it can be said that if $g(y)$ is irreducible over Q, then $f(x)$ is also irreducible over Q. Hence, the polynomial $x^4 + x^3 + x^2 + x + 1$ is irreducible over Q.\end{proof}

\end{document}